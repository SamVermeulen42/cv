\documentclass[a4paper,11pt,oneside]{article}
\usepackage{a4wide}                     % Iets meer tekst op een bladzijde
\usepackage[dutch]{babel}               % Voor nederlandstalige hyphenatie (woordsplitsing)
\usepackage{amsmath}                    % Uitgebreide wiskundige mogelijkheden
\usepackage{amssymb}                    % Voor speciale symbolen zoals de verzameling Z, R...
\usepackage{url}                        % Om url's te verwerken
\usepackage{graphicx}                   % Om figuren te kunnen verwerken
\usepackage[small,bf,hang]{caption2}    % Om de captions wat te verbeteren
\usepackage{xspace}                     % Magische spaties na een commando
\usepackage[latin1]{inputenc}           % Om niet ascii karakters rechtstreeks te kunnen typen
\usepackage{float}                      % Om nieuwe float environments aan te maken. Ook optie H!
\usepackage{flafter}                    % Opdat floats niet zouden voorsteken
\usepackage{listings}                   % Voor het weergeven van letterlijke text en codelistings
\usepackage{marvosym}                   % Om het euro symbool te krijgen
\usepackage{textcomp}                   % Voor onder andere graden celsius
\usepackage{fancyhdr}                   % Voor fancy headers en footers.
\usepackage{graphics}			% Om figuren te verwerken.
\newcommand{\npar}{\par \vspace{2.3ex plus 0.3ex minus 0.3ex} \noindent}	% Om witruimte te krijgen tussen paragrafen
\hyphenation{stu-den-ten-ver-te-gen-woor-di-ger op-lei-dings-com-mis-sies com-pu-ter-com-mis-sies fa-cul-teits-raad In-ge-nieurs-We-ten-schap-pen}

%opening
\title{Curriculum Vitae}
\author{Sam Vermeulen}

\begin{document}

\maketitle

\section{Personalia}
\begin{itemize}
  \item Naam: Sam Vermeulen
  \item Adres: Margrietstraat 97, 9170 Meerdonk
  \item Telefoon: 0498/25.33.28
  \item E-mail: sam.verm@yahoo.com
  \item Geboortedatum: 15 maart 1993
	\item Hobby's en interesses:
		\begin{itemize}
		\item Muziek: ik speel bugel
		\item Strategische bordspellen met vrienden
		\item Ik ben ook iemand die graag enkele kleine hobby projecten maakt voor vrienden die er nood aan hebben of uit persoonlijk interesse. Voorbeelden hiervan zijn de basis van een android game, een volledig rolluiken systeem, een betaalsysteem voor de Chiro en een programma om save files van Visual Led Pois aan te passen.
		\end{itemize}
\end{itemize}

\section{Opleiding}
\begin{itemize}
\item 2015-heden: Burgerlijk Ingenieur in de Computerwetenschappen, Universiteit Gent
\item 2011-2015: Informatica, Universiteit Gent 
\item 2005-2011: Industri�le Wetenschappen, GTI Beveren
\end{itemize}

\section{Studentenervaringen}
\begin{itemize}
\item Stages
	\begin{itemize}
	\item zomer 2016: stage bij Amplidata Gent: creatie van een systeem voor de analyse van logfiles van een groot aantal teams in 1 uniform overzicht
	\end{itemize}
\item Vakantiejobs
	\begin{itemize}
	\item zomer 2009-2015: jobstudent: arbeider bij Stora Enso Lumipaper, Kallo
	\end{itemize}
\item Extracurriculaire activiteiten
	\begin{itemize}
	\item 2016-2017: ICT verantwoordelijke Home Astrid: voornamelijk troubleshooting van it-problemen bij homebewoners
	\end{itemize}
\end{itemize}

\section{Vaardigheden}
\subsection{ICT-vaardigheden}
	\begin{itemize}
	\item Systeembeheer: Windows, Linux
	\item Programmeren: Java, C(++), Python, Haskell, Prolog
	\item Algemeen: netwerken en netwerkbeheer, beveiliging (zowel software als netwerken), software ontwerp, databases, besturingssystemen, gedistribueerde systemen
	\end{itemize}

\subsection{Kantoor}
	\begin{itemize}
	\item Ervaren met Linux (Ubuntu voornamelijk), Windows, OpenOffice, LaTeX, web, e-mail, Github, Jira en Jenkins
	\end{itemize}

\subsection{Talenkennis}
	Mijn moedertaal is Nederlands en beheers ik dan ook uitstekend. Engels beheers ik redelijk goed en kan dan ook op alle vlakken vlot communiceren. Frans beheers ik in mindere mate maar kan ik wel begrijpen.

\subsection{Rijbewijs}
	\begin{itemize}
	\item Houder van een Europees rijbewijs B
	\end{itemize}

\end{document}
